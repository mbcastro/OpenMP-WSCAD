\documentclass{SBCbookchapter}
\usepackage[utf8]{inputenc}
\usepackage[T1]{fontenc}
\usepackage[brazilian]{babel}
\usepackage{graphicx}

\author{Márcio Castro, Pedro H. Penna e Alyson D. Pereira\\
\textit{Departamento de Informática e Estatística (INE)}\\
\textit{Universidade Federal de Santa Catarina (UFSC)}
}
\title{Desenvolvimento de Aplicações Paralelas Eficientes com OpenMP}

\begin{document}
\maketitle

\begin{abstract}
\end{abstract}

%\begin{resumo}
%\begin{otherlanguage}{brazilian}
%\end{otherlanguage}
%\end{resumo}



\section{Introdução}

O capítulo apresentará uma motivação e introdução ao uso da API de programação OpenMP para o desenvolvimento de aplicações paralelas para arquiteturas de memória compartilhada.

\section{Regiões Paralelas e o Modelo Fork-Join}

Esse capítulo irá abordar os conceitos básicos de programação paralela com OpenMP. Primeiramente será apresentado o modelo base do OpenMP (Modelo Fork-Join). Em seguida, será apresentada a primitiva básica para a criação de regiões paralelas no código (\texttt{omp parallel}). Por fim, o serão apresentadas as formas de compartilhamento e privatização de dados oferecidas pelo OpenMP (\texttt{private}, \texttt{firstprivate} e \texttt{shared}).

\section{Paralelismo de Dados e Diretivas OpenMP}

O paralelismo de dados será explorado através do estudo das diretivas de compilação disponíveis no OpenMP para paralelização de laços (\texttt{omp for} e \texttt{omp parallel for}). Além disso, serão discutidas as diferentes estratégias de escalonamento disponíveis no ambiente de execução OpenMP (\texttt{static}, \texttt{dynamic} e \texttt{guided}) assim como as suas vantages e desvantagens. Por fim, será apresentada a cláusula de redução (\texttt{reduction}) juntamente com os seus casos de uso.

\section{Paralelismo de Tarefas e Diretivas OpenMP}

O paralelismo de tarefas será explorado através do estudo das diretivas de compilação disponíveis no OpenMP para paralelização de trechos de códigos. Serão discutidas duas construções: seções (\texttt{omp sections}) e tarefas (\texttt{omp task} e \texttt{omp taskwait}). A utilização de tarefas será abordada com o auxilio de problemas recursivos clássicos.

\section{Sincronização}

Nesse capítulo serão discutidas as diretivas de sincronização de threads disponíveis no OpenMP: serialização de código (\texttt{omp single} e \texttt{omp master}), exclusão mútua (\texttt{omp critical}) e barreiras (\texttt{omp barrier}).

\section{Conclusão}

Esse capítulo apresentará um apanhado geral do minicurso, ressaltando os principais pontos abordados no texto.

\end{document}
